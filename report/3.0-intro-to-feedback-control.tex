\chapter{Introduction to feedback control}
Control is something we deal with on a daily basis. Whether it is the temperature in our houses, the number of people in line at the supermarket's checkout or something more critical like the number of neutrons emitted during a nuclear reaction in a reactor, we have to control it to avoid potential chaos.

In software development we too want to have control over the products we both create and use. For example, most applications have a settings menu where the user can alter the application's behavior to their liking; to ensure our code to be working correctly we use compilers, IDEs and automated testing tools that help us trace errors and bugs; companies like Google, Facebook and Twitter do A/B-testing and other kinds of user studies when introducing new features and with that control and improve the usability of their product; and last but not least, in cloud computing we want to control the number jobs in a queue and scale up or down depending on the amount of jobs in the queue, such that all jobs get executed as fast as possible with spending the least amount of resources.

A common factor in controlling a system is the notion of feedback: you change something in the environment and see how well it adjusts towards your desired outcome. If the lines at checkout get too long, most likely an extra cash register needs to be opened; if your users don't perform well on an A/B-test of your latest features, you should probably revise these features rather than bringing them into production; and when you alter the settings in an application, you check whether the new behavior is more to your desire and maybe fiddle around with them some more until it is more to your liking.

Controlling a system without the notion of feedback is possible, however this requires an exact model of the system under control. External forces that are not part of the model cannot be allowed to disturb the system. In most cases this model does unfortunately not exist and control has to incorporate the system's output to come up with it's next input. If the output is not taken into account, the system can drift off and end up in undesired situations. A nice example of this are Microsoft's user studies: the user's feedback was not taken into account when Office 2007 introduced the ``ribbon'' to replace the existing user interface nor when the start button disappeared in Windows 8 and suddenly every Windows user lost their 20+ years of accumulated muscle memory for commanding the Windows operating system \cite{meijer2014-embracing-the-hacker-way}.

In physics and engineering, feedback control is a commonly used technique and is also fully backed by mathematics. A series of equations and theorems such as Laplace Transforms and differential equations describe the abstract behavior of a feedback controlled system and allow to evaluate its outcome after running it for a certain time and with a certain input \cite{hellerstein2004-feedback, janert2013-feedback}. These theorems and equations are usable in science because everything else is described in the `language of mathematics' as well. We have known the differential equations to Newton's law of cooling, damped harmonic oscillators and so forth for centuries, as well as their Laplace Transforms, which are needed for the `feedback equations'.

This is all in contrast to the current situation in computer science. Although proven to be very effective in physics, computer science has not yet widely adopted the techniques of feedback control. Instead, complex algorithms are written to control a web cache or do cloud scaling. Although it seems very naturally applicable, it is surprising to find that such a simple and effective technique is mostly ignored.

A potential reason for this lack of using feedback control lies in the fact that computer science does not yet fully understand the whole mathematical background of the datastructures and applications that are being used and created. Given this lack of understanding, we are not yet able to describe to which laws our datastructures and application obey. What would be the equivalent of Newton's laws for a web cache \cite{janert2013-feedback}? Only recently some work has been done on understanding the mathematics behind this \cite{beckmann2015-cache-calculus} and we may very well see more of this in the future, which will eventually enable computer science to create a more formal mathematical model with theorems and equations equivalent to physics.

The lack of applying feedback control in computer science is also reflected in this technique commonly not being taught in university degree computer science education. Although it is one of the standard courses in physics and engineering studies, it is not at all present in the computer science curricula. And in the rare occasions it is taught, it is done in a way that a physics student would learn about it, namely as yet another maths course, even though this way of treating feedback control in computer science is not really applicable!

In this chapter we will introduce the basic concepts of feedback control with as little mathematics as possible. We will start with an overview of what a feedback system consists of and how it works. Next we will go into some detail on how to control a feedback system. Finally we will introduce a simple toy example that demonstrates the power of feedback control and translates these concepts into an imperative style program.

\section{Basics of feedback control}
In general a system is controlled by feedback when its next input value is (partially) determined by its previous output(s). The reason for taking the previous output into account while coming up with a next input is due to external forces that may impacting the system in unexpected ways. These external forces do not come regularly, are not predictable and are also not equally strong each time, hence a notion of uncertainty in the system's behavior occurs. Due to this uncertainty, a model that accurately calculates the system's next input either does not exist or would be too complex. The solution for incorporating the uncertainty from external forces is to use a feedback cycle around the system: the system's output (affected by both the system's input and any external forces) is measured and compared to a desired reference value, after which their difference is transformed into the system's next input.

Important to mention here early on is that controlling a system with a feedback cycle is not a solution to optimization problems. Tasks like ``\textit{Make the flow through the system as large as possible}'' cannot be accomplished with feedback loops \cite{janert2013-feedback}, as they compare the system's current output with a known reference value. Tasks like this require some kind of optimization strategy that determines the reference value, after which a feedback loop can be used to bring and keep the system's output in this desired state.

The input and output of a feedback controlled system are not to be confused with the actual input and output of this system. The air conditioning or heating system has an actual output of hot or cool air, whereas the feedback system's output can be any other measurable metric such as the new temperature after the actual output is applied or the temperature difference caused by the actual output. The input and output of a feedback controlled system are often referred to \textit{control input} and \textit{control output}.

\subsection{Calculating the next control input}
Every time the control output produces a new value, it is compared to a reference value or \textit{setpoint} for it to calculate the \textit{tracking error} as the deviation of the control output from the setpoint:

\begin{equation} \label{eq:tracking-error}
\text{tracking error} = \text{setpoint} - \text{control output}
\end{equation}

This tracking error is then transformed into the next control input by the \textit{controller}. When the tracking error is positive (thus the control output is less than the setpoint), the controller has to produce a new control input that ultimately raises the control output to the same level as the setpoint, such that the tracking error becomes zero. Of course this depends on the \textit{directionality} of the system: for some systems the control input needs to be lowered for the system output to be raised. The controller must be able to make this distinction and therefore has to know the directionality of the system. For example, a heating system requires the controller to \emph{raise} the control input to get a higher temperature, whereas a cooling system requires the controller to \emph{lower} the control input to get a higher temperature.

Besides the directionality, the controller also needs to decide on the magnitude of the correction. If the magnitude is too high, the controller could overcompensate and turn a positive tracking error into a negative tracking error and vice versa, causing the system to oscillate between two states. The worst situation occurs when the negative tracking error caused by the overcompensation of the positive tracking error is larger than the positive tracking error: this results in an ever growing amplitude of the oscillation, which eventually makes the whole system unstable and in the end causes it to blow up.

On the other hand, the magnitude can be too low, causing the controller to undercompensate. This causes tracking errors to persist for a longer time than necessary and makes the system respond slow to disturbances. Although this is less dangerous than instability, this slow behavior is unsatisfactory as well.

In general we require the controller to take in a tracking error and come up with a new control input such that the tracking error will go to zero as soon as possible. For this it turns out that the controller does not need to know anything about the controlled system, but only requires information about the directionality of the system and the magnitude of the correction.

\subsection{Architectural overview}
The architecture of a feedback system is usually depicted as a set of boxes connected with arrows. This way we get a quick overview of the design without worrying about the exact implementation of the various components. The general architecture of a feedback system as discussed above is shown in \Cref{fig:feedbackArchitecture}. Notice that here the control output is negated on the way back and is then added to the setpoint in order to calculate the tracking error. This is common practice as some more preprocessing is required before comparing the control output with the setpoint. For example, the output may contain noise which needs to be smoothened by some kind of filter. Of course preprocessing steps will also be drawn in this overview if applicable.

\begin{figure}[H]
	\begin{center}
		\includegraphics[width=0.85\textwidth]{figures/FeedbackBig.png}
	\end{center}
	\caption{The architecture of a feedback control system}
	\label{fig:feedbackArchitecture}
\end{figure}

Besides these filters, a controller may consist of multiple components by itself. When using an incremental controller, only the difference in the control input is outputted. For the actual input we need to maintain a running sum of all the previous controller outputs and in that way calculate the actual control inputs. Usually these extra steps are also depicted in the architectural overview as extra boxes and arrows.


\section{Different types of controllers}
The control input in a feedback system is the value that tells the system under control to behave slightly different than it did and possibly incorporating external disturbances on the system in a better way. As discussed in the previous section, this control input value is computed in the controller part of the feedback system. Although one might expect differently, the controller itself does not turn out to be very smart or know a whole lot about the system under control. As mentioned before, a controller in fact only needs to know about directionality of the system and the magnitude of the correction for it to work just fine. There are a number of controllers that only need these two pieces of information and that are commonly used in physics, mechanics and electronics.

\subsection{On/Off control}
The most simple controller one can think of is just an on/off switch. Whenever the tracking error is positive, the controlled system is turned on and when the tracking error becomes negative it is turned off again\footnote{Of course this is dependent upon the directionality of the system under control.}. For simple systems this kind of control will suffice, although it will not be a very effective approach.

An application of this kind of control might be a air conditioning system that turns on when the temperature exceeds a preset level. In this example the control output is the current temperature in the room, the setpoint is the preset temperature, the control input is a boolean value which determines whether the system should be on of off and the directionality of the system is negated. Imagine the temperature initially being much too high, causing the air conditioning to turn on right away. After a certain amount of time the temperature reaches the desired setpoint (the tracking error becomes zero), hence the system will shut down. Shortly after the air conditioning system is shut down, the temperature starts increasing again. This immediately causes a deviation from the setpoint, forcing the air conditioning to turn on again. Soon enough the temperature is low enough again for the system to be turned off, after which the cycle starts all over again. Of course this kind of behavior is very annoying to everyone working in this room, as the air conditioning continuously turns off and on again. Besides that, this behavior costs an unnecessary amount of energy for turning the system on and off, which is not really desirable either.

Obviously this behavior is caused by the controller that dictates to turn off the system whenever the setpoint is met. Due to external disturbances such as whether conditions the feedback system is off track soon again, which causes the controller to decide to turn the controlled system back on.

Small improvements that are often used in these kinds of systems are introducing a dead zone or Schmitt trigger, which causes the system to continue with the same corrective action until a certain threshold or a certain amount of time is exceeded. This prevents the controller from overreacting on sudden and short spikes in the control output. In the case of the air conditioning an addition such as this will cause the controller to not immediately send a `turn off signal' when the setpoint is zero and will not activate the system with the slightest deviation, but instead waits a little longer until a threshold is exceeded before activating or deactivating the system again.

\subsection{Proportional control}
Another simple controller 


\section{An extended example}
To get a better feel for how a feedback control system works in practice, we will discuss a simple but interesting application of feedback control that uses the theory covered in the previous sections. In this section we show an imperative reference implementation. In the next chapter we will continue to use and refactor this application as we come up with an API for constructing and executing feedback systems. The application at hand is a port from the original implementation by Nikita Leshenko in Javascript, CSS and HTML \cite{nikital-balltracker}. We will however use Scala as our programming language and JavaFx for drawing the graphics on the screen.

The application consists of a flat surface on which a ball can move around. The goal is to move the ball from its initial position to the position on the surface that the user clicks on with the mouse. \autoref{fig:balltracker-initial} shows the ball in its initial state when the application starts. When the user clicks on the screen, the ball starts moving to that position as shown by \autoref{fig:balltracker-moving}. Besides the background and the ball, also the desired position, a dashed line between the ball and the desired position, the acceleration in both the $x$ and $y$ directions and a trail of previous positions are drawn (which are fading away over time). After some time the ball is at its desired position and waits there for a new position to navigate to (\autoref{fig:balltracker-reached}). Notice that the desired position can also be updated while the ball is still moving, in which case it moves to the new and discards the old destination.

In order to allow the ball to move in a natural way, we remind the reader of some basic equations from physics that describe the one dimensional motion of an object as a function of time (\autoref{eq:motion}). Here $x_t$, $v_t$ and $a_t$ are the ball's position, velocity and acceleration at time $t$ respectively. For two-dimensional motion we can combine two sets of these equations for both dimensions.

\begin{subequations}
	\begin{equation}
		x_t = x_{t - 1} + v_t \cdot \Delta t
	\end{equation}
	\begin{equation}
		v_t = v_{t - 1} + a_t \cdot \Delta t
	\end{equation}
	\label{eq:motion}
\end{subequations}

\begin{figure}
	\centering
	\begin{subfigure}[b]{0.80\linewidth}
		\includegraphics[trim = 0mm 110mm 0mm 0mm, clip, width=\linewidth]{figures/BallTracker-initial.png}
		\caption{Initial}
		\label{fig:balltracker-initial}
	\end{subfigure}
	
	\begin{subfigure}[b]{0.80\linewidth}
		\includegraphics[trim = 0mm 110mm 0mm 0mm, clip, width=\linewidth]{figures/BallTracker-moving.png}
		\caption{Moving to desired position}
		\label{fig:balltracker-moving}
	\end{subfigure}
	
	\begin{subfigure}[b]{0.80\linewidth}
		\includegraphics[trim = 0mm 110mm 0mm 0mm, clip, width=\linewidth]{figures/BallTracker-overshooting.png}
		\caption{Moving to desired position}
		\label{fig:balltracker-moving}
	\end{subfigure}
	
	\begin{subfigure}[b]{0.80\linewidth}
		\includegraphics[trim = 0mm 110mm 0mm 0mm, clip, width=\linewidth]{figures/BallTracker-desired.png}
		\caption{Reached desired position}
		\label{fig:balltracker-reached}
	\end{subfigure}
	\caption{Ball tracker}
	\label{fig:balltracker}
\end{figure}

We define these required pieces of physics in \autoref{lst:ball-physics}. We first define a \code{Position}, \code{Velocity} and \code{Acceleration} as tuples of \code{Double} as well as mathematical operations on the tuple type. The \code{Ball} class describes the current state of the ball having a position, velocity and acceleration. A second constructor (\code{apply}) defines the initial position with no acceleration or velocity. The method \code{accelerate} on this class takes a new acceleration and calculates the new state for the ball according to \autoref{eq:motion}. Notice that we discard the $\Delta t$ term as this is always equal to 1. To keep track of the ball's previous positions we define a type \code{History}, being a queue of \code{Position}s.

\begin{minipage}{\linewidth}
\begin{lstlisting}[style=ScalaStyle, caption={Ball motion physics}, label={lst:ball-physics}]
type Position $=$ (Double, Double)
type Velocity $=$ (Double, Double)
type Acceleration $=$ (Double, Double)
type History $=$ mutable.Queue[Position]

implicit class Tuple2Math[X: Numeric, Y: Numeric](val src: (X, Y)) {
  import Numeric.Implicits._
  def +(other: (X, Y)) $=$ (src._1 + other._1, src._2 + other._2)
  def -(other: (X, Y)) $=$ (src._1 - other._1, src._2 - other._2)
  def *[Z](scalar: Double) $=$ (src._1.toDouble * scalar, src._2.toDouble * scalar)
  def map[Z](f: X $\Rightarrow$ Z)(implicit ev: Y $=:=$ X): (Z, Z) $=$ {
    ((x: X, y: Y) $\Rightarrow$ (f(x), f.compose(ev)(y))).tupled(src)
  }
}

case class Ball(acc: Acceleration, vel: Velocity, pos: Position) {
  def accelerate(newAcc: Acceleration): Ball $=$ {
    Ball(newAcc, vel + newAcc, pos + vel + newAcc)
  }
}
object Ball {
  def apply(radius: Double): Ball $=$ {
    Ball((0.0, 0.0), (0.0, 0.0), (radius, radius))
  }
}
\end{lstlisting}
\end{minipage}

\todo{text here}

%\begin{minipage}{\linewidth}
\begin{lstlisting}[style=ScalaStyle, caption={Ball drawing}, label={lst:ball-drawing}]
object Draw {
  def draw(pos: Position, setpoint: Position, acc: Acceleration, history: History)(implicit gc: GraphicsContext) $=$ {
    drawBackground
    drawHistory(history)
    drawLine(pos, setpoint)
    drawSetpoint(setpoint)
    drawBall(pos)
    drawVectors(pos, acc)
  }

  def drawBackground(implicit gc: GraphicsContext) $=$ {
    gc.setFill(Color.rgb(231, 212, 146))
    gc.fillRect(0, 0, width, height)
  }

  def drawBall(point: Position)(implicit gc: GraphicsContext) $=$ {
    val (x, y) $=$ point
    val diameter $=$ 2 * ballRadius

    gc.setFill(Color.rgb(123, 87, 71))
    gc.fillOval(x - ballRadius, y - ballRadius, diameter, diameter)
  }

  def drawSetpoint(setpoint: Position)(implicit gc: GraphicsContext) $=$ {
    val (x, y) $=$ setpoint
    val radius $=$ ballRadius / 4
    val diameter $=$ radius * 2

    gc.setFill(Color.rgb(161, 90, 90))
    gc.fillOval(x - radius, y - radius, diameter, diameter)
  }

  def drawLine(ball: Position, setpoint: Position)(implicit gc: GraphicsContext) $=$ {
    val (bx, by) $=$ ball
    val (sx, sy) $=$ setpoint

    gc.setStroke(Color.rgb(96, 185, 154))
    gc.setLineWidth(1.0)
    gc.setLineDashes(8.0, 14.0)

    gc.beginPath()
    gc.moveTo(sx, sy)
    gc.lineTo(bx, by)
    gc.stroke()
    gc.setLineDashes()
  }

  def drawVectors(pos: Position, acc: Acceleration)(implicit gc: GraphicsContext) $=$ {
    val (px, py) $=$ pos
    val (ax, ay) $=$ acc

    gc.setStroke(Color.rgb(247, 120, 37))
    gc.setLineWidth(8)
    gc.setLineCap(StrokeLineCap.ROUND)

    gc.beginPath()
    gc.moveTo(px, py)
    gc.lineTo(px - ax * 300, py)
    gc.stroke()

    gc.beginPath()
    gc.lineTo(px, py)
    gc.lineTo(px, py - ay * 300)
    gc.stroke()
  }

  def drawHistory(history: History)(implicit gc: GraphicsContext) $=$ {
    history.synchronized {
      history.zipWithIndex.foreach(item $\Rightarrow$ {
        val (pos, index) $=$ item
        val (x, y) $=$ pos
        val size $=$ history.size
        val alpha $=$ (index: Double) / size

        gc.setFill(Color.rgb(96, 185, 154, alpha))
        gc.fillOval(x, y, 10, 10)
      })
    }
  }
}
\end{lstlisting}
%\end{minipage}
