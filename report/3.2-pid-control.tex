\section{Different types of controllers}
The control input in a feedback system is the value that tells the system under control to behave slightly different than it did and possibly incorporating external disturbances on the system in a better way. As discussed in the previous section, this control input value is computed in the controller part of the feedback system. Although one might expect differently, the controller itself does not turn out to be very smart or know a whole lot about the system under control. As mentioned before, a controller in fact only needs to know about directionality of the system and the magnitude of the correction for it to work just fine. There are a number of controllers that only need these two pieces of information and that are commonly used in physics, mechanics and electronics.

\subsection{On/Off control}
The most simple controller one can think of is just an on/off switch. Whenever the tracking error is positive, the controlled system is turned on and when the tracking error becomes negative it is turned off again\footnote{Of course this is dependent upon the directionality of the system under control.}. For simple systems this kind of control will suffice, although it will not be a very effective approach.

An application of this kind of control might be a air conditioning system that turns on when the temperature exceeds a preset level. In this example the control output is the current temperature in the room, the setpoint is the preset temperature, the control input is a boolean value which determines whether the system should be on of off and the directionality of the system is negated. Imagine the temperature initially being much too high, causing the air conditioning to turn on right away. After a certain amount of time the temperature reaches the desired setpoint (the tracking error becomes zero), hence the system will shut down. Shortly after the air conditioning system is shut down, the temperature starts increasing again. This immediately causes a deviation from the setpoint, forcing the air conditioning to turn on again. Soon enough the temperature is low enough again for the system to be turned off, after which the cycle starts all over again. Of course this kind of behavior is very annoying to everyone working in this room, as the air conditioning continuously turns off and on again. Besides that, this behavior costs an unnecessary amount of energy for turning the system on and off, which is not really desirable either.

Obviously this behavior is caused by the controller that dictates to turn off the system whenever the setpoint is met. Due to external disturbances such as whether conditions the feedback system is off track soon again, which causes the controller to decide to turn the controlled system back on.

Small improvements that are often used in these kinds of systems are introducing a dead zone or Schmitt trigger, which causes the system to continue with the same corrective action until a certain threshold or a certain amount of time is exceeded. This prevents the controller from overreacting on sudden and short spikes in the control output. In the case of the air conditioning an addition such as this will cause the controller to not immediately send a `turn off signal' when the setpoint is zero and will not activate the system with the slightest deviation, but instead waits a little longer until a threshold is exceeded before activating or deactivating the system again.

\subsection{Proportional control}
Another improvement on the on/off controller is to change the output. Rather than sending a boolean value to the controlled system, a real controller should send a floating point value. The proportional controller does this by taking the magnitude of the error into account when calculating the magnitude of the corrective action. If the tracking error is small, the control input should be small and if a large tracking error occurs, the control input should be proportionally large. To achieve this, the proportional controller uses the following formula:

\begin{equation} \label{eq:proportional-control}
u_p(t) = k_p \cdot e(t) \text{\ \ \ \ where } k_p > 0 \text{ constant}
\end{equation} 

Here $u_p(t)$ is the control input at time $t$, $e(t)$ is the tracking error at time $t$ and $k_p$ is the proportional controller gain, which is a positive constant.